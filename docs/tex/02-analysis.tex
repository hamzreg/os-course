\chapter{Аналитический раздел}

\section{Постановка задачи}

В соответствии с заданием на курсовую работу необходимо разработать загружаемый модуль ядра для мониторинга состояния физических страниц, выделенных процессу.

Для решения поставленной задачи необходимо:

\begin{itemize}
	\item проанализировать структуры и функции ядра, предоставляющие информацию о физических страницах, выделенных процессу;
	\item разработать алгоритмы и структуру программного обеспечения;
	\item реализовать программное обеспечение;
	\item провести анализ работы разработанного программного обеспечения.
\end{itemize}

\section{Анализ управления памятью}

Операционная система Linux является системой с поддержкой виртуальной памяти \cite{love}. Для каждого процесса создается адресное пространство, которое делится на блоки равного размера, называемые страницами. Размер страницы определяется системно. Единицей деления физической памяти является физическая страница или страничный кадр. При таком механизме организации памяти каждой виртуальной странице ставится в соответствие физический адрес.

Для отображения виртуальной памяти в физическую используются таблицы страницы. Виртуальный адрес разбивается на части: индексы и смещение. Информация, хранящаяся в элементе таблицы страниц, соответствующем заданному индексу, указывает либо на адрес другой таблицы,  либо на адрес физической страницы памяти, в которой хранятся нужные данные. Схема преобразования виртуального адреса в физический адрес показана на рисунке \ref{img:address-translation}.

\begin{figure}[H]
	\begin{center}
		\includegraphics[scale=0.75]{inc/img/transformation.pdf}
	\end{center}
	\captionsetup{justification=centering}
	\caption{Преобразование виртуального адреса в физический адрес}
	\label{img:address-translation}
\end{figure}
    
Для каждого процесса создается адресное пространство, которое описывается структурой, называемой дескриптором памяти. В поле \texttt{pgd} дескриптора памяти хранится указатель на глобальный каталог страниц текущего процесса \texttt{PGD}. Элементы в таблице \texttt{PGD} содержат указатели на каталоги страниц среднего уровня \texttt{PMD}. Элементы таблиц \texttt{PMD} содержат указатели на таблицы \texttt{PTE}, записи которой указывают на страницы физической памяти. Связь таблиц страниц представлена на рисунке \ref{img:page-tables}.

\begin{figure}[H]
	\begin{center}
		\includegraphics[scale=0.75]{inc/img/page-tables.pdf}
	\end{center}
	\captionsetup{justification=centering}
	\caption{Связь таблиц страниц}
	\label{img:page-tables}
\end{figure}

\section{Анализ структур ядра, предоставляющих информацию о страницах, выделенных процессу}

Для описания процесса в ядре используется структура \texttt{task\_struct}, которая называется дескриптором процесса. Структура описана в файле \texttt{<linux/sched.h>}. Дескриптор процесса позволяет получить информацию о состоянии процесса, открытых файлах и другом. Некоторые поля этой структуры приведены в листинге \ref{lst:task_struct.c}.

\includelistingpretty
    {task_struct.c}
    {C}
    {Структура task\_struct}
    
У объекта \texttt{task\_struct} есть поле \texttt{mm}, содержащее указатель на структуру \texttt{mm\_struct} --- дескриптор памяти процесса. Эта структура описана в файле \texttt{<linux/mm\_types.h>}. Дескриптор памяти процесса предоставляет всю информацию, относящуюся к адресному пространству процесса. Некоторые поля этой структуры показаны в листинге \ref{lst:mm_struct.c}.

\newpage

\includelistingpretty
    {mm_struct.c}
    {C}
    {Структура mm\_struct}
    
Поле mmap объекта \texttt{mm\_struct} содержит указатель на структуру \texttt{vm\_area\_struct}, которая определена в файле \texttt{<linux/mm\_types.h>}. Данная структура используется для описания одной непрерывной области памяти в данном адресном пространстве. Некоторые поля этой структуры представлены в листинге \ref{lst:vm_area_struct.c}.
    
\includelistingpretty
    {vm_area_struct.c}
    {C}
    {Структура vm\_area\_struct}

Каждый дескриптор памяти связан с уникальным диапазоном адресов в адресном пространстве процесса. В поле \texttt{vm\_start} \texttt{vm\_area\_struct} хранится начальный адрес этого диапазона, а в поле \texttt{vm\_end} --- адрес первого байта, расположенного после описываемого диапазона.

Для описания физической страницы памяти в ядре используется структура \texttt{page}, которая определена в файле \texttt{<linux/mm\_types.h>}. Некоторые поля этой структуры приведены в листинге \ref{lst:page.c}.

\includelistingpretty
    {page.c}
    {C}
    {Структура page}

Для страниц, которые можно сопоставить с пользовательским пространством, в поле \texttt{\_mapcount} объекта \texttt{page} хранится количество записей в таблице страниц, указывающих на данную страницу \cite{oracle}.

В поле \texttt{\_refcount} объекта page хранится счетчик использования страницы, отражающий количество ссылок в системе на эту страницу. Только что выделенная страница имеет счетчик ссылок, равный 1. Когда счетчик ссылок достигает нуля, страница освобождается.

Для определения счетчика ссылок на страницу используется функция ядра \texttt{page\_ref\_count}, определенная в \texttt{<linux/page\_ref.h>} (листинг \ref{lst:page_ref_count.c}):

\includelistingpretty
    {page_ref_count.c}
    {C}
    {Определение счетчика ссылок на страницу}
    
\section{Анализ использования записей таблицы страниц}

Для доступа к дескриптору физической страницы используются записи таблицы страниц \cite{tables}. Для перемещения по таблицам страниц в файле \texttt{<asm/pgtable.h>} определены следующие функции ядра (листинг \ref{lst:walk_tables.c}):

\includelistingpretty
    {walk_tables.c}
    {C}
    {Перемещение по таблицам страниц}
    
Доступ к физической странице осуществляется следующим образом:

\begin{enumerate}
	\item Системный вызов \texttt{pgd\_offset()} принимает дескриптор памяти процесса и виртуальный адрес и возвращает элемент глобального каталога страниц \texttt{PGD} текущего процесса.
	\item Системный вызов \texttt{pmd\_offset()} запись глобального каталога страниц \texttt{PGD} и виртуальный адрес и возвращает запись каталога страниц среднего уровня \texttt{PMD}.
	\item Системный вызов \texttt{pte\_offset\_map()} принимает элемент каталога страниц среднего уровня \texttt{PMD} и виртуальный адрес и возвращает запись таблицы \texttt{PTE}.
	\item Системный вызов \texttt{pte\_page()} по полученной записи таблицы \texttt{PTE} возвращает объект \texttt{page}.
\end{enumerate}

Функции ядра, представленные в листинге \ref{lst:funcs_node.c}, используются для определения того, существуют ли соответствующие записи таблицы страниц. В случае отсутствия записи в таблице возвращается 1.
    
\includelistingpretty
    {funcs_node.c}
    {C}
    {Определения счетчика ссылок на страницу}
    
Функции ядра, показанные в листинге \ref{lst:funcs_bad.c}, используются для проверки записей при передаче в качестве входных параметров функциям, которые могут изменить значение записей.
    
\includelistingpretty
    {funcs_bad.c}
    {C}
    {Определения счетчика ссылок на страницу}

\section*{Вывод}

В соответствии с проведенным анализом для решения поставленной задачи необходимо разработать алгоритм сканирования виртуальных страниц и алгоритм обращения к дескриптору физической страницы. Для доступа к дескриптору физической страницы следует использовать глобальный каталог страниц \texttt{PGD}, каталог страниц среднего уровня \texttt{PMD} и таблицу страниц \texttt{PTE} процесса. Для доступа к таблицам страниц процесса необходимо обращаться к следующим структурам ядра: \texttt{task\_struct}, \texttt{mm\_struct}, \texttt{vm\_area\_struct}. Для получения информации о состоянии физической страницы следует обращаться к полям структуры \texttt{page}.
