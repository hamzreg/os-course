\chapter{Технологический раздел}

\section{Выбор языка и среды программирования}

В качестве языка программирования был выбран язык C \cite{c}, так как большая часть исходного кода ядра Linux, его модулей и драйверов написана на C.

В качестве среды разработки была выбрана Visual Studio Code \cite{vscode} в связи с простотой и удобством использования.

\section{Функции загрузки и выгрузки модуля}

При загрузке модуль ядра получает идентификатор процесса, с использованием которого находится дескриптор процесса. Если идентификатор процесса не будет задан, мониторинг не будет произведен, в системный журнал будет выведено соответствующее сообщение.

Функции загрузки и выгрузки модуля показаны в листинге \ref{lst:load-unload.c}.

\includelistingpretty
    {load-unload.c}
    {C}
    {Функции загрузки и выгрузки модуля}
    
\section{Функция сканирования виртуальных страниц}

Алгоритм сканирования виртуальных страниц представлен в листинге~\ref{lst:scan_virtual_pages.c}.

\includelistingpretty
    {scan_virtual_pages.c}
    {C}
    {Функция сканирования виртуальных страниц}
    
\newpage

\section{Функция обращения к дескриптору физической страницы}

Алгоритм функции обращения к дескриптору физической страницы приведен в листинге \ref{lst:walk_page_table.c}.

\includelistingpretty
    {walk_page_table.c}
    {C}
    {Функция обращения к дескриптору физической страницы}
    
\section{Функция получения информации о состоянии физической страницы}

Алгоритм функции получения информации о состоянии физической страницы показан в листинге \ref{lst:log_page.c}.

\includelistingpretty
    {log_page.c}
    {C}
    {Функция получения информации о состоянии физической страницы}