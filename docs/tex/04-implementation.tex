\chapter{Технологический раздел}

\section{Выбор языка и среды программирования}

В качестве языка программирования был выбран язык C \cite{c}, так как большая часть исходного кода ядра Linux, его модулей и драйверов написана на данном языке.

В качестве среды разработки была выбрана Visual Studio Code \cite{vscode}.

\section{Реализация загружаемого модуля ядра}

\subsection{Функции загрузки и выгрузки модуля}

При загрузке модуль ядра получает идентификатор процесса, с использованием которого находится дескриптор процесса. Если идентификатор процесса не будет задан, мониторинг не будет произведен, в системный журнал будет выведено соответствующее сообщение.

Реализация функций загрузки и выгрузки модуля представлены в листинге \ref{lst:load-unload.c}.

\includelistingpretty
    {load-unload.c}
    {C}
    {Функции загрузки и выгрузки модуля}
    
\subsection{Функция прохода по виртуальным адресам}

Реализация функции прохода по виртуальным адресам показана в листинге \ref{lst:walk_addresses.c}.

\includelistingpretty
    {walk_addresses.c}
    {C}
    {Функция прохода по виртуальным адресам}
    
\newpage
    
\subsection{Функция получения физической страницы}

Реализация функции получения физической страницы приведена в листинге \ref{lst:get_page.c}.

\includelistingpretty
    {get_page.c}
    {C}
    {Функция получения физической страницы}
    
\subsection{Функция получения информации о состоянии страниц, выделенных процессу}

Реализация функции получения информации о состоянии страниц, выделенных процессу показана в листинге \ref{lst:log_page.c}.

\includelistingpretty
    {log_page.c}
    {C}
    {Функция получения информации о состоянии страниц, выделенных процессу}